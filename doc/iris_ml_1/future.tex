We studied the Root Cause Analysis (RCA) problem for integrity error diagnosis that is critical to reliably operate large-scale distributed applications. The main challenges of incomplete network state information, low probabilistic nature of gray failures, and infeasible scalable system model have hindered the development of efficient diagnostic technologies and tools for systems of realistic scales. Without the need for accurate system models, emerging machine learning techniques have inspired much research interest in network system RCA recently, which mainly focused on data center networks with known topology and routing information.

In this study, we targeted the integrity error RCA for large-scale distributed application systems deployed over a wide area multi-domain network environment. We formulated this problem as a multi-class classification problem. Our main hypothesis that the mapping between the application level flow characteristics and the network component failures for RCA can be learned with high accuracy without the network topological and routing information due to the technical and administrative barriers. 

We built a high-fidelity emulation environment in a cloud testbed that allows creating virtual systems to emulate arbitrary large-scale networks and run the real application software. We further developed a suite of software tools to automate the whole analysis workflow that includes virtual network creation, routing control plane configuration, integrity error injection, experimental data transfer, integrity check, and training data collection. This environment makes experiments highly repeatable and reproducible and the trained models have the potential to be directly used in the production systems being emulated.

We then conducted extensive RCA ML model analysis with the harvested data from emulating a data transfer workflow system over a wide-area multi-domain network with many end hosts. For inference, we defined the network-wide aggregated data flow as the input and a Top-k accuracy metric based on class probability distribution, which can boost the inference accuracy significantly. We specifically quantified the impacts of training data features that include the flow coverage between complete of partial set of end hosts and flow attributes. We further looked into the inherited training data imbalance problem due to the differentiated failure probabilities in network and storage components. Different combinations of these factors reflect different data availability scenarios in realistic system settings. They also introduce mixed numerical and categorical features and data imbalance in the training data that may affect the ML model choice and tuning.  Among several multi-class classification models we evaluated, the random forest model demonstrated the best performance.

For our future work, we plan to experiment with networks of larger scale with different graph characteristics using finer tuned ML models. We are also working on integrating our system with the advanced monitoring and workflow applications towards realtime RCA in production systems.

