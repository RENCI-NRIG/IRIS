For large-scale distributed applications involving intensive data transfers over a network, root cause analysis (RCA) for error diagnosis becomes extremely challenging. One main reason is that the underneath network is often of multi-domain nature, where limited component level information is available, and where it is not possible to instrument and measure traffic flows at all the nodes in the network. It's hard, if not completely possible, to build proper system diagnosis models. As a result, RCA remains a guessing art that requires manual debugging and daunting amount of communication between operators from different domains and subsystems that often takes days or even weeks.  

Among the more challenging failure modes are the so-called ``grey failures" that only come to effect randomly, often at low probability. One such common failure mode is the integrity errors that may corrupt data, which is caused either by the storage subsystem or a network component along data transfer path. 

In this paper, we focus on the integrity error RCA problem in a networked system setting. We first cast it as a multi-label classification problem with the goal to enable root cause inference for a given network only from the flow level check at the end hosts. Collecting sufficient training data is fundamentally difficult with a machine learning approach, which is typical for production systems due to lack of monitoring services. We therefore built an emulation environment in a Cloud testbed that allows creating arbitrary large-scale virtual network systems. We further developed a suite of software tools to automate the training process that includes configuring the network routing, instrumenting data transfers between end hosts, injecting arbitrary integrity errors into the network components, and collecting and processing the raw data. This gives us extra benefits in experiment repeatability and efficiency. For evaluation, we trained several multi-label classification models and validated their performance with a Top-$k$ accuracy metric for an emulated network. The results demonstrate the efficacy of the approach and potential to apply the learned models to production systems.  
