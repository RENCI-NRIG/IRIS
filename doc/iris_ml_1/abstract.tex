In this paper, we present a system to localize the root causes of integrity errors in large-scale networked systems. 
This particular ``gray failure mode" has raised much research interest recently 
because it can cause severe data corruptions but the faulty system element(s) 
can stay stealth from the existing reliable protocols like TCP, encrypted transfer and RAID, etc.

Most existing RCA systems take an infrastructure operator's view and rely on a dedicated packet level monitoring infrastructure.
In contrast, we take an application-centric view and target a typical 
wide area network environment where complete network and packet monitoring information are normally lacking, but end-to-end flow monitoring information 
can be instrumented and obtained from the applications. We thereafter adopt a multi-class classification machine learning (ML) approach to design a system to learn the mappings 
between the application level flow measurement and the network component failures.

We built a high-fidelity emulation environment in a cloud testbed with a suite of software tools that can 
emulate data transfer workflows over arbitrary wide-area network topologies and inject integrity errors at chosen network elements.
This environment serves two purposes. It can generate training data sets for us to evaluate our system under 
realistic network conditions and different feature availabilities. It could also be used as a sandbox to train a high-fidelity 
RCA model for a production network system to use.
   
We demonstrated that root cause localization with high accuracy can be obtained with a random forest model and data balancing techniques. 
We especially studied the impacts of different realistic combinations of features based on the available information from the measurement system.   
