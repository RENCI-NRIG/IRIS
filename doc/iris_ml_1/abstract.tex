For large-scale distributed applications involving intensive data transfers over a network, root cause analysis (RCA) for error diagnosis becomes extremely challenging. One main reason is that the underneath network is often of multi-domain nature, where limited component level information is available, and where it is not possible to gather sufficient measurements. It is also challenging to build comprehensive system diagnosis models. As a result, RCA remains a guessing art that requires intensive manual debugging and daunting amount of communication between operators from different organizations that often takes days or even weeks.  

Among the more challenging failure modes are the so-called ``grey failures" that only come to effect randomly, often at low probability. One such common failure mode is the integrity errors that may corrupt data, which is caused either by the storage subsystem or a network component along data transfer path. 

In this paper, we first cast the integrity error RCA problem as a multi-class classification problem with the goal to enable root cause inference only from the flow level check at the end hosts. Collecting sufficient training data is fundamentally difficult with a machine learning approach, which is typical for production systems due to lack of monitoring services and administrative access. We therefore built an emulation environment in a Cloud testbed that allows creating virtual systems to mimic arbitrary large-scale networks and run real application software. We further developed a suite of software tools to automate the training process that includes configuring the network routing, instrumenting data transfers between end hosts, injecting arbitrary integrity errors into the network components, and collecting and processing the raw data. This gives us extra benefits in experiment repeatability and efficiency. 

We conducted extensive experiments on a network in above emulation environment to generate training data and detailed data analysis to gain insights in three aspects that are critical to our RCA 
problem: (1) performance comparison in terms of accuracy and training time performance of different models, (2) impact quantification of mixed numerical and categorical features on the model performance, 
(3) handling of the inherited data imbalance in the training data set. Specifically to the network RCA problem, we used network-wide data aggregation to define data samples for inference and a Top-$k$ accuracy 
metric based on class probability distribution to significantly improve the model performance. The results demonstrate the efficacy of the approach and potential to extend to production systems.  
