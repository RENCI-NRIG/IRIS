Due to the scalability challenge, some recent work in large scale data center networks (possibly thousands of nodes) adopted stochastic learning approaches in localizing tomography based performance downgrades or probabilistic grey failure issues. Basically, they attempted to learn an estimate regression model between the output variables in the end-to-end performance measurements and the root causes as the input variables in either performance degradation at the end hosts~\cite{NetPoirot:Sigcomm2016} or gray failure probabilities inside the network nodes or links~\cite{Link-JIoT-2019,netbouncer:nsdi18}. Both inputs and outputs are explicitly defined as continuous variables. Their technical challenges are the regression model optimization algorithms (e.g., regularization and gradient methods) and statistical significance test~\cite{DeepView:NSDI18}. 

Our targeted networks are of multi-domain nature where the RCA granularity can be limited to individual domains instead of individual routers inside a domain, which are totally unknown to the outside. The network can be represented as a simple graph $G(V,E)$ where $V$ is the set of nodes that includes $H$ end hosts and $R$ routers. $E$ is the set of links. We say a file is corrupted when it incurs integrity errors. In general, the problem at hand can be concisely represented by the following formula.

\begin{flalign}\label{eq:prob}
\begin{aligned}
P(File\ i\ succeeds) =\\
f(F_i, \prod_{j \in Path\ i}P(component\ j\ does\ not\ cause\ errors) )
\end{aligned}
\end{flalign}

As our main concern is if a file (flow) is corrupted rather than packet losses, the file characteristics, $F_i$, eg, the file size, transfer time, etc. may play an important role. We emphasize again that the components ($j$) of $Path\ i$ that file transfer $i$ traverses is unknown. And we need large amount of data transfer flows to generate sufficient training data since our targeted grey failure, the integrity error, has very low probability (often in the order of $10^{-3}$).

We model our problem as a multi-label classification problem where the labels are defined as all the nodes and links that may incur integrity errors and the features are flow level characteristics that include source, destination, size, transfer time, throughput, whether a flow is corrupted, missed, or retried, etc. In general, the training process takes as input two arrays: an array X of size $[n_{samples}, n_{features}]$ holding the training samples, and an array y of class labels of size $[n_{samples}]$. The total number of labels is the number of the links and the end hosts in the topology, $L=|E|+|H|$.

As there is a large number of different models and associated parameters to be tuned, in this paper, we will not try to exhaust all the models and extensive parameter tuning. Rather, we choose the following three supervised learning models that have proved suitable for multi-class classification in the literature. We use the popular Scikit-learn library to implement these methods and used the default parameters in training these models. 

{\bf Random Forests.}  Decision tree is a natural choice to multi-class classification as the multiple leaves represent the labels. Its $predict\_proba$ method gives the class membership probability estimates. One of its main advantages is the fast prediction time after the tree is trained. In this study, we used the ensemble method based on randomized decision trees. By averaging over multiple base decision trees built from randomized samples and making the splitting decisions in each tree with randomized subset of features, the random forests model may efficiently reduce the overfitting. 

{\bf Support Vector Machine (SVM).} When using SVM for multi-class classification, the ``one-against-one" approach is adapted. As such, the training may take long time to converge when the data set is big or the feature set is big. The $decision\_function$ method of SVC (Support Vector Classification) produces per-class scores for each sample which also enables the class membership probability estimates. We experimented with both linear SVC and the general SVC with the default RBF kernel.

{\bf Bayesian Networks (BN).} Since our ultimate goal is to infer the cause of the failures, BN is a model that is worth investigating. Specifically, we use the Multinomial Naive Bayes method, which is suitable for the multi-class classification. Again its $predict\_proba$ method gives the class membership probability estimates.

\subsection{Data unit definition and training accuracy.}
One key observation is that an erratic link may cause integrity errors on all paths traversing it. While the training can be done with the set of individual flows, inference is better to be done in the unit of all aggregated flows that are affected by a particular cause, i.e., a particular label. As a result, the training accuracy should be computed with the same unit of aggregated flows. As there are $L$ labels, all the labeled data (one per flow) will be aggregated into $M$ instances to be tested against the trained model. The resulting total number of correct label matches divided by $L$ is defined as the accuracy.

\subsection{Top-$k$ classification accuracy.} 
Since we assume training data from data transfer flows only between the end hosts, it doesn't satisfy the necessary condition presented in~\cite{netbouncer:nsdi18}. The conventional classification on a single label will perform poorly. The top-k classification can be very useful in practice as it produces a small set of highly likely causes for the operators to zoom in. Top-$k$ accuracy means that any of the labels of $k$ highest probability in the classification results matching the correct label is true positive.