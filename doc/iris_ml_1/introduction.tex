Root cause analysis (RCA) is a critical function in operating and managing complex networked systems, be it physical, software, or hybrid~\cite{RCA-Review-2017}.
It aims to identify the component(s) and process(es) responsible for the fault manifested by the wrong results or system failures in a timely fashion.
Traditional RCA relies on system domain models that can be used to deduce the potential component failures from the system symptoms and behaviors.
However, with the exponential increase of system scales, the complex of component interdependencies, and the lack of visibility in multi-domain, large-scale networks have 
significantly made it harder to build such models for efficient fault identification and localization for modern distributed systems. 
As a result, RCA using machine learning (ML) or data analytic approaches has drawn extensive research attention and has found prominent uses
 in data center networks and Internet applications~\cite{netbouncer:nsdi18,Link-JIoT-2019,microrca:noms2020}. 

In this paper, we take a new application domain, the scientific workflow management systems (WMS), as our primary motivating use case to tackle the complex
 system RCA problem using a machine learning approach. WMS facilitates in-order execution of jobs in workflows and includes large amounts of interdependent
  data transfers, storage functions, and computation tasks. These tasks are often distributed over distributed hardware, software, and data resources
   located in different facilities nationwide or globally. Inevitably, frequent system failures and reliability issues 
caused by errors and faults from underlying subsystems have been serious concerns for the WMS community. 

Since a typical workflow system runs as a middleware sitting several layers above the infrastructure resources that are managed 
and operated by different service providers (domains), it has a limited view of the health of the infrastructure components, the exact topology of the network and the routing path of the data movement,
sometimes even the sources and sinks are normally abstracted in virtual namespaces. Consequently, root cause localization of the failures become extremely difficult and normally cost coordinated efforts and long hours 
from many operators of different domains. 

In a nutshell, ML-based RCA can be formulated as a multi-class classification problem, where the potential root causes are the labels and various 
measurement and observations of the tasks and data flow level observations are used as the features.
RCA for large-scale networked systems poses other unique challenges in adopting ML approaches.
 
The first challenge comes from the so-called {\it gray failures} in the networked system. Due to their probabilistic nature, the faulty components would act normally most of the time and therefore are hard to be caught. Recent studies showed that {\it gray failures} causing performance degradation in terms of packet losses and latency could be efficiently localized using the ML approach~\cite{GrayFailure:2017,DeepView:NSDI18}.  The data integrity errors, on the other hand, may randomly corrupt bits in a block of data or packets over network transfer. Since existing checksum mechanisms implemented in TCP and the storage services are not sufficient to guarantee end-to-end data integrity, they often get unnoticed for a long time until severe consequences to applications occurred. Modern middleware systems have just started to add end-to-end integrity check mechanisms, including in Pegasus~\cite{swip:pearc:2019} and Globus~\cite{IntegrityVerification:DataTransfer}. However, root cause localization of integrity errors remains an unsolved problem.

The second challenge lies in the difficulties in acquiring sufficient training data to an ML-based RCA system. For a system over the Internet, in addition to passive monitoring data, active probes or event fault injections are often used to generate more diagnosis data~\cite{active:iot:2019, NetPoirot:Sigcomm2016}. 

The last but not the least challenge is to design and train the right ML models to achieve high diagnosis accuracy, out of a large pool of candidates~\cite{Boutaba:2018aa}. We will show that even the most basic questions of defining a data point, feature selection, and performance evaluation need special consideration.

In~\cite{Link-JIoT-2019}, the authors attempted to identify the stop failure of network links via the popular multi-class ML models using end-to-end passive traffic engineering measurements (throughput, latency, and packet loss). The authors in \cite{DeepView:NSDI18} took an active probe approach to localize the fault in a virtual disk system to the finest granularity up to the network switches. In~\cite{netbouncer:nsdi18}, a necessary condition was derived on the minimal set of paths that active probes need to be sent over the targeted network. Another line of work including~\cite{NetPoirot:Sigcomm2016,KDD14} adapted a statistical learning approach to infer the probabilistic relationship between the path failure and the link faults. All these research works made a strong assumption that one can instrument probes or observations from any node to any other node in the network since their target systems are data center networks that they own. In an earlier study, the decision tree model was used to predict if a request will succeed over a flawed network system~\cite{DT:2004}. Bayesian inference was demonstrated to be efficient for fast diagnosis when the causal relationship model is established in a large Internet system~\cite{BN-Internet:2007}.

In summary, the majority of existing studies took an infrastructure operators' view in which they focused on using packet-level statistics and assumed availability of packet routing and component-level monitoring capabilities. 
In contrast, in this paper, we take an overlay view that is more realistic in the targeted multi-domain environment where routing and monitoring are often lacked for application service providers. Based on this view, we developed 
an efficient ML-based RCA system that could fill the gap between the infrastructure and application system in failure diagnosis and localization.

The rest of this paper is organized as follows. We first define network system integrity error RCA problem in Section~\ref{sec:integrity}. We present a high-fidelity system emulation environment in a cloud testbed in Section~\ref{sec:emulation}. We then describe the machine learning models in Section~\ref{sec:ml}. Evaluation results are presented in Section~\ref{sec:evaluation} and the paper is concluded in Section~\ref{sec:future}.
