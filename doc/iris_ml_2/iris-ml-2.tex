\documentclass[conference]{IEEEtran}
\usepackage{graphicx}
\usepackage{amsmath}
\usepackage{amssymb}
\usepackage{subfig}
\usepackage{wrapfig}

\usepackage[font=small,labelfont=bf]{caption}

\usepackage{array}
\newcolumntype{P}[1]{>{\centering\arraybackslash}p{#1}}
\newcolumntype{M}[1]{>{\centering\arraybackslash}m{#1}}

\newcommand{\ie}{{\em i.e.}}
\newcommand{\eg}{{\em e.g.}}
\newcommand{\et}{{\em et al.~}}
\newcommand{\real}{\mathbb{R}}
\newcommand{\integer}{\mathbb{N}}

\newtheorem{theorem}{Theorem}
\newtheorem{lemma}{Lemma}
\newtheorem{corollary}{Corollary}
\newtheorem{definition}{Definition}

\DeclareMathOperator*{\argmax}{arg\,max}
\DeclareMathOperator*{\argmin}{arg\,min}

\renewcommand{\refname}{\centerline{References cited}}
\newcommand{\required}[1]{\section*{\hfil #1\hfil}}

% this handles hanging indents for publications
\def\rrr#1\\{\par
\medskip\hbox{\vbox{\parindent=2em\hsize=6.12in
\hangindent=4em\hangafter=1#1}}}

\def\baselinestretch{1}

\begin{document}

\title{Root Cause Analysis of Data Integrity Errors in Networked Systems with Incomplete Information
\thanks{This work was supported by the
    US National Science Foundation under Grant OAC-1839900.}
}
\author{\IEEEauthorblockN{Yufeng Xin, Shih-Wen Fu, \\ Anirban Mandal, Ilya Baldin }
\IEEEauthorblockA{%RENCI\\
RENCI, UNC - Chapel Hill\\
Chapel Hill, NC, USA\\
}
\and
\IEEEauthorblockN{Ryan Tanaka, Mats Rynge, \\Karan Vahi, Ewa Deelman}
\IEEEauthorblockA{%ISI\\
ISI, USC\\
Marina Del Rey, CA, USA\\
}
\and
\IEEEauthorblockN{Ishan Abhinit, Von Welch}
\IEEEauthorblockA{%CACR\\
CACR, Indiana University\\
Bloomington, IN, USA\\
}
}

\maketitle
\thispagestyle{empty}

\begin{abstract}
For large-scale distributed applications involving intensive data transfers over a network, root cause analysis (RCA) for error diagnosis becomes extremely challenging. One main reason is that the underneath network is often of multi-domain nature, where limited component level information is available, and where it is not possible to gather sufficient measurements. It is also challenging to build comprehensive system diagnosis models. As a result, RCA remains a guessing art that requires intensive manual debugging and daunting amount of communication between operators from different organizations that often takes days or even weeks.  

Among the more challenging failure modes are the so-called ``grey failures" that only come to effect randomly, often at low probability. One such common failure mode is the integrity errors that may corrupt data, which is caused either by the storage subsystem or a network component along data transfer path. 

In this paper, we first cast the integrity error RCA problem as a multi-class classification problem with the goal to enable root cause inference only from the flow level check at the end hosts. Collecting sufficient training data is fundamentally difficult with a machine learning approach, which is typical for production systems due to lack of monitoring services and administrative access. We therefore built an emulation environment in a Cloud testbed that allows creating virtual systems to mimic arbitrary large-scale networks and run real application software. We further developed a suite of software tools to automate the training process that includes configuring the network routing, instrumenting data transfers between end hosts, injecting arbitrary integrity errors into the network components, and collecting and processing the raw data. This gives us extra benefits in experiment repeatability and efficiency. 

We conducted extensive experiments on a network in above emulation environment to generate training data and detailed data analysis to gain insights in three aspects that are critical to our RCA 
problem: (1) performance comparison in terms of accuracy and training time performance of different models, (2) impact quantification of mixed numerical and categorical features on the model performance, 
(3) handling of the inherited data imbalance in the training data set. Specifically to the network RCA problem, we used network-wide data aggregation to define data samples for inference and a Top-$k$ accuracy 
metric based on class probability distribution to significantly improve the model performance. The results demonstrate the efficacy of the approach and potential to extend to production systems.  

\end{abstract}

\section{Introduction}
\label{sec:introduction}
Root cause analysis (RCA) is a critical function in operating and managing complex networked systems, be it physical, software, or hybrid~\cite{RCA-Review-2017}.
It aims to identify the component(s) and process(es) responsible for the fault manifested by the wrong results or system failures in a timely fashion.
Traditional RCA relies on system domain models that can be used to deduce the potential component failures from the system symptoms and behaviors.
However, with the exponential increase of system scales, the complex of component interdependencies, and the lack of visibility in multi-domain, large-scale networks have 
significantly made it harder to build such models for efficient fault identification and localization for modern distributed systems. 
As a result, RCA in such a distributed and opaque system setting has drawn extensive research attention in recent years, which have found prominent uses
 in data center networks and Internet applications. Not surprisingly, these studies have adopted the machine learning (ML) or data analytic approaches
due to the relaxed requirements on accurate domain models~\cite{netbouncer:nsdi18,Link-JIoT-2019,microrca:noms2020}.

In this paper, we take a new application domain, the scientific workflow management systems (WMS), as our primary motivating use case to tackle the complex
 system RCA problem using a machine learning approach. WMS facilitates in-order execution of jobs in workflows and includes large amounts of interdependent
  data transfers, storage functions, and computation tasks. These tasks are often distributed over distributed hardware, software, and data resources
   located in different facilities nationwide or globally. Inevitably, frequent system failures and reliability issues 
caused by errors and faults from underlying subsystems have been serious concerns for the WMS community. 
Therefore most WMS have built-in failure handling mechanisms like redundancy and automatic retries. 
They also try to provide as much log information as possible to help with failure diagnosis, which normally assumes a long manual process.

Since a typical workflow system runs as a middleware sitting several layers above the infrastructure resources that are managed 
and operated by different service providers (domains), it has a limited view of the health of the infrastructure components. 
Most critically, it has no direct knowledge of the exact topology of the network and the routing path of the data movement,
sometimes even the sources and sinks are normally abstracted in virtual namespaces.  
Consequently, fault diagnosis and root cause analysis of the failures become extremely difficult and normally cost coordinated efforts and long hours from many operators of different sub-systems. Those failures often include many unsuccessful (and wasted) retries from the users. 
Hence, analyzing the root cause of failures for data integrity errors in distributed workflow executions is a representative yet very difficult problem.

In a nutshell, ML-based RCA can be formulated as a multi-class classification problem, where the potential root causes are the labels and various 
measurement and observations of the tasks and data flow level observations are used as the features.
RCA for large-scale networked systems poses other unique challenges in adopting ML approaches.
 
The first challenge comes from the prevalence of so-called {\it gray failures} in the networked system, in addition to the normal {\it stop failures}~\cite{GrayFailure:2017,DeepView:NSDI18}. A {\it stop failure} is a kind of hard failure, which refers to the complete breakdown of a component that disrupts all traffic flows or paths over this component deterministically~\cite{Link-JIoT-2019}. {\it Gray failures} are those probabilistic failures in a component that would act normally most of the time and that could not be caught by the traditional deterministic system monitoring and diagnostic tools. We categorize two types of  {\it gray failures}: performance degradation and data integrity errors. The former is normally caused by system overload that will lead to slow response, timeout, and the frequent reboot of the servers or software. The latter may randomly corrupt bits in a block of data or packets over network transfer. Since existing checksum mechanisms implemented in TCP and the storage services are not sufficient to guarantee end-to-end data integrity, they often get unnoticed for a long time until severe consequences to applications occurred. Therefore modern WMS have started to add end-to-end integrity check mechanisms, including in Pegasus~\cite{swip:pearc:2019} and Globus~\cite{IntegrityVerification:DataTransfer}.

The second challenge lies in the difficulties in acquiring sufficient training data feed to an ML-based RCA system. For a system over the Internet, the global routing information is not completely available for the RCA system as they are normally proprietary to different service providers (domains). The possible monitoring data sources or active probe sites in a network are always limited. As a result, in addition to passive monitoring data, active probes or event fault injections are often used to generate more diagnosis data~\cite{active:iot:2019, NetPoirot:Sigcomm2016}. Then, due to the desire of conducting RCA in real-time, how to minimize the overhead and latency of data collection in a production setting becomes a significant design issue.

The last but not the least challenge is to design and train the right ML models to achieve high diagnosis accuracy, out of a large pool of candidates~\cite{Boutaba:2018aa}. We will show that even the most basic questions of defining a data point, feature selection, and performance evaluation need special consideration.

In~\cite{Link-JIoT-2019}, the authors attempted to identify the stop failure of network links via the popular multi-class ML models using end-to-end passive traffic engineering measurements (throughput, latency, and packet loss). The authors in \cite{DeepView:NSDI18} took an active probe approach to localize the fault in a virtual disk system to the finest granularity up to the network switches. In~\cite{netbouncer:nsdi18}, a necessary condition was derived on the minimal set of paths that active probes need to be sent over the targeted network. Another line of work including~\cite{NetPoirot:Sigcomm2016,KDD14} adapted a statistical learning approach to infer the probabilistic relationship between the path failure and the link faults. All these research works made a strong assumption that one can instrument probes or observations from any node to any other node in the network since their target systems are data center networks that they own. In an earlier study, the decision tree model was used to predict if a request will fail or succeed over a flawed network system~\cite{DT:2004}. Bayesian inference was demonstrated to be efficient for fast diagnosis when the causal relationship model is established in a large Internet system~\cite{BN-Internet:2007}.

In this paper, we first cast the WMS integrity error diagnosis as a networked system Root Cause Analysis (RCA) problem, where data transfers only occur between end hosts and no flow routing and network monitoring information is available (Section~\ref{sec:integrity}). Our only assumptions on the network underlay are that the elements in the network are known and the file integrity at the end hosts of data transfers can be checked. In order to obtain sufficient training data and make experiments efficient and repeatable, we created an experimental system in a cloud testbed that can automatically create a large-scale OSPF-enabled virtual network system, initiate data transfers between end hosts, and inject arbitrary integrity errors into the virtual router interfaces and end hosts (Section~\ref{sec:emulation}). We then studied several variants from three different families of multi-class classification models, Bayesian inference, SVM, and Decision Tree. We identified that using the network-wide aggregated data flow as the input and a Top-$k$ accuracy metric can significantly improve the inference performance (Section~\ref{sec:ml}). With the data collected from the emulation, we evaluated the model performance in Section~\ref{sec:evaluation}. We specifically quantified the impacts of training data coverage in terms of total flows between all or part of the end hosts, mixed numerical and categorical features, and the inherited data imbalance in the training data set. For the emulated network system, the results show that a random forest model with the right set of training data and inference method can precisely localize the root cause of integrity errors to the single network interface or end host. The analysis validated our main hypothesis that the mapping between the application level flow characteristics and the network component failures for RCA can be learned and inferred with high accuracy from a sufficiently large amount of labeled training data. We conclude the paper in Section~\ref{sec:future}.


\section{A Network Failure Localization Model}
\label{sec:fault}
In this section, we present the inference model to localize the failures based on the path level measurements. 
Fig.~\ref{fig:example} shows a simple example to illustrate the problem. This networked system consists of several sites interconnected by a network cloud 
where only the network nodes and their interfaces are known (otherwise the failure localization is meaningless). 
A distributed application will incur traffic flows along an unknown path between pairs of end hosts. These flows are subject to 
the measurement of the applications to test if they are corrupted. Two such flows, $c3-d2$ and $c2-d4$ are shown in the figure. Intuitively, 
if both flows suffer from data corruption, the interface with the cross mark in Node $N2$ should be inferred to be the culprit.   

\begin{figure}
  \begin{center}
    \includegraphics[width=0.48\textwidth]{./figure/example_network.png}
  \end{center}
\caption{An Example Network}
\label{fig:example}
\end{figure}

In most existing work on in gray failure localization, the network is modeled as a simple graph $G(V,E)$  
with a set of nodes $V$ connected by a set of links $E$. The following bipartite mapping 
formula was used to capture the relationship between the link failures and the path level measurements~\cite{netbouncer:nsdi18,DeepView:NSDI18,arzani2018democratically}. 
The failure being considered is packet loss. 
The reasonings behind these models are similar: due to scalability or privacy constraints, 
monitoring every component of interest in a large-scale network is 
not feasible, while the path level measurement is more practical to deploy and instrument.  
\begin{flalign}\label{eq:prob}
\begin{aligned}
&P(No\ Failure\ in\ Path \ i) = \\
&\prod_{j \in Path\ i}P(component\ j\ is\ normal)
\end{aligned}
\end{flalign}
This can be transformed to a familiar linear regression model for every identifiable path in the network after taking log on the equations.
\begin{flalign}\label{eq:linear}
\begin{aligned}
&p_i = &\sum_{j \in Path\ i} c_j\ & \forall i \in P
\end{aligned}
\end{flalign}
Here $P$ represents the set of paths that are measured and a fundamental assumption underscoring these models is that routing of every path 
in $P$ over link set $E$ needs to be obtained. In addition, it was also assumed the measurement system have the access to all network nodes 
to instrument path measurements, \ie, the source and destination of a path can be any nodes. In summary, to 
establish the regression model to obtain satisfactory inference performance, 
substantial efforts were made to (i) identify the routing of the paths, (ii) determine the path set for good coverage, and 
(iii) enable constant measurement of paths. Then the path measurements ($p_i$) will be obtained to estimate the link error probability ($c_j$) using this model.

In our wide-area multi-domain network setting, as we discussed in an earlier work~\cite{iris:ictc21}, it is not practical to identify the routing of 
the paths over the network and even the network topology beforehand. This means that it is difficult to establish a model like Eq. (\ref{eq:linear}). 
We also can not assume access to nodes in the network domains to instrument or measure paths.  

We thereafter distinguish between application end hosts (that generate and receive data) and networking devices (routers or switches), 
\ie, $V$ includes $H$ end hosts and $R$ routers. And we redefine $E$ to be the set of network components where failures are supposed to be 
localized, specifically all the network interfaces on $V$ and the end hosts $H$. We further constrain that only passive path measurements are available from certain applications on the end hosts, \ie, $P$ in our system only consists of paths originating from and ending in end hosts in $H$, which implies a much smaller identifiable path set. In the example network Fig.~\ref{fig:example}, none of the network nodes, $N1, \ldots, N6$, can be the source or destination of a path. And for a path between two end hosts, its routing is unknown. The only knowledge our model has is the bag of nodes and their interfaces for traffic forwarding.  

We further observe that one component failure (e.g., $x_j$) 
could cause multiple paths erroneous while one erroneous path may be the result of a failure at different components. The standard model Eq. (\ref{eq:linear}) ignores 
the correlation between multiple paths sharing a common component. We thereafter inverse the equations to represent the component failure probability as a function of 
the path failure probabilities as the following prediction model. 
 
\begin{flalign}\label{eq:inverse}
\begin{aligned}
Y = F(x_1, \cdots, x_p, \cdots, x_{|P|} ) \\
 = \sum_{p \in P} w_p x_p +w_0
\end{aligned}
\end{flalign}

Specifically, $Y$ represents a vector space $(y_1, \ldots, y_v, \ldots, y_{|V|)}$ where $y_v$ represents the failure probability of component 
$v \in V$.  $X = (x_1, \ldots, x_p, \ldots, x_{|P|})$  forms the feature space that is defined by the combinations of the path failure probability. 
As shown in Eq. (\ref{eq:inverse}), we can further make it a linear regression model, which produces excellent performance as we will show in 
the evaluation section. We note, unlike in the existing work where failure localization is on network links, the network components in our model 
are the nodes and their interfaces because we assume the network topology is unknown.

Since any component failure only affects a small number of paths that go through it, plus multiple simultaneous failures are rare in reality, it is 
reasonable to expect both the feature matrix and the coefficient matrix is sparse, representing the samples collected during one inference 
window. This suggests using the regularization technique to make most of the estimated coefficient to be zero. The most efficient technique 
to achieve this intention is to add a L1-norm constraint is known as Lasso~\cite{DeepView:NSDI18}, where the regression optimization objective 
is defined as:    

\begin{flalign}\label{eq:lasso}
\begin{aligned}
\hat{W} =  \argmin_{W \in R^{|P|}}\vert\vert{\textbf{Y}-\textbf{X}W}\vert\vert _2^2 + \lambda \vert\vert{W}\vert\vert_1
\end{aligned}
\end{flalign}

Here $\textbf{Y}$ and $\textbf{X}$ are the sample matrix. This technique has proven extremely efficient in dealing with overfitting. 
The vector definition of $Y \in R^{|C|}$ means for each sample $n$, all entries but one in $Y^n$ are zeros. 
Compared to the scalar variable of a specific component failure probability, this multi-output model 
captures the independence between all the failures and would help the training and prediction quality.  










\section{Missing Data and Imputation}
\label{sec:sl}
As we discussed in Section~\ref{sec:introduction}, missing data is pervasive in reality due to lost or unavailable measurement data. 
It means that some samples, in the training set or the test set, have missing features. 
Using the example network in Fig.~\ref{fig:example} to illustrate, during a diagnosis time window, the application may not incur traffic between 
$c1$ and $d1$, or it never needs to transfer data between the origin sites, or the application measurement system may corrupt or lose some 
measurement data for some traffic flow. All these will lead to 'holes' in the feature columns in the data sets. In the first two cases, 
entire feature columns will be missing in our model~\ref{eq:inverse}.

Missing data can be categorized into three types: (i) the data is missing completely at random (MCAR) if the missingness does not depend 
on any of the observed and unobserved variables, (ii) the data is missing at random (MAR) if the missingness is dependent only on the observed variables, 
(iii) the data is missing not at random (MNAR) if the missingness is neither MCAR nor MAR, \ie, the missingness depends on both observed variables and the 
unobserved variables. The majority of existing studies used the MCAR assumption~\cite{Yoon2018GAINMD}. 

There are many kinds of missing data recovery methods commonly used in the literature. These methods largely fall into three categories.

The {\it univariate} methods impute values in a  feature dimension using only non-missing values in that feature dimension. It simply replaces the missed 
values with certain statistics of the non-missing values such as the zero, mean, median, mode, max, or min. 

The {\it multivariate} imputation algorithms use the entire set of available feature dimensions to estimate the missing values based on the assumption 
of correlations between the feature dimensions. Each feature with missing values is modeled as a function of other features, and therefore the imputation itself 
is modeled as a regression problem that is trained and used to estimate the imputation. In order to achieve the best performance, especially to avoid the 
overfitting from certain features, it is conducted in a series of regression iterations: at each step, a feature is used as the output of other features and the resulted model is used to estimate the missing feature. After all features are processed or the designated max iteration is reached, the results of the final estimation are used to impute 
the missing data.

Our prediction model in Eq.~(\ref{eq:inverse})  uses the path (flow) measurements as the input. It naturally fits the multivariate imputation approach 
because the path failures caused by a common component failure are correlated. In contrast, the existing models based on Eq.~(\ref{eq:linear}) use the 
component failure as the input variables that are independent of each other. The {\it multivariate} imputation does not seem to make sense.
The {\it univariate} method is deemed not applicable due to the sparse nature of the feature matrix and lack of reasonable explanation.

Most recently,  the Generative Adversarial Nets ({\it GAN)} framework has shown good performance to generate the missing data.
In this model, the generator’s goal is to accurately impute missing data, and the discriminator’s goal is to distinguish between observed and imputed 
components. The discriminator is trained to minimize the classification loss (when classifying which components were observed and 
which have been imputed), and the generator is trained to maximize the discriminator’s misclassification rate. Thus, these 
two networks are trained using an adversarial process~\cite{Yoon2018GAINMD,Awan2021ImputationOM}.

There are off-the-shelf libraries that support both univariate and multivariate imputations in popular software packages like R and 
Scikit-learn~\cite{JSSv045i03,10.1371/journal.pone.0254720}. The imputation can also be performed multiple times with different 
random number seeds to generate multiple imputations. This is important if the statistical analysis is needed, \eg, in the medical domain. 

Most of existing missing data studies focus on minimizing the imputation errors of the data in the feature space. However the ultimate goal 
is the performance of the prediction models after missing data is imputed.

Corresponding to our model in Eq.~(\ref{eq:inverse}), missing data will cause values of some $x_p$ to be null. 
The feature space is defined in a $|P|$-dimensional space $\mathbf{X} = \mathbf{X_1} \times \ldots \times \mathbf{X_{|P|}}$. 
Following the MCAR assumption 
on the missing data, we can define a mask vector $M = (M_1, \ldots, M_{|P|})$ taking random values in ${(0, 1)}^{|P|}$.  
A sample vector $X = (X_1 \times \ldots \times X_{|P|})$ 
can be masked by $M$ to generate a corresponding sample vector with missing data $\tilde{X} = \tilde{X}_1 \times \ldots \times \tilde{X}_{|P|}$ as follow:

\[
\tilde{X_p} = 
\begin{cases}
  X_p & \text{if $M_p = 1$} \\
  null & \text{otherwise}
\end{cases}
\]

From an arbitrary missing rate $r \in (0, 1)$, a random mask vector $M_r$ can be created to emulate missing data from a given feature matrix $X$. 
For a particular missing feature $X_r \in X$, the imputation essentially creates a regression model that makes $X_r$ the output variable and all the 
other features the input variables.   

\begin{flalign}\label{eq:imputation}
\begin{aligned}
X_r = F(x_1, \cdots, x_p, \cdots, x_{|P|} ), \ p \neq r \\
\end{aligned}
\end{flalign}

At the end of the imputation, a recovered data set $\hat{X_r}$ is generated. The goal is to make these as close as possible.

Our main results are based on the MCAR missing data model, the {\it multivariate} imputation algorithms, and regularized regression model, 
which can be summarized in the following pipeline definition with Scikit\_Learn.
%~\cite{Scikit:web}. 
\begin{verbatim}
    estimator = make_pipeline(
        IterativeImputer(random_state=0, 
        		missing_values=np.nan, 
        		estimator=impute_estimator),
        PolynomialFeatures(poly),
        br_estimator
    )
\end{verbatim}

In the pipeline, the {\it impute\_estimator} specifies the regressor for missing data imputation and the {\it br\_estimator} specifies the regressor to infer the 
localized failure probability. We added a {PolynomialFeatures} element to evaluate if polynomials of higher degree perform better than the linear regressor.

This pipeline construct allows us to systematically evaluate the performance of multiple regressors in both {\it impute\_estimator} and {\it br\_estimator}, as well 
as tuning their hyperparameters. As we discussed earlier, in theory, Lasso should be a suitable regressor in both places. We also evaluated other popular 
regressors that include Ridge, BayesianRidge, ExtraTreesRegressor, and KNeighborsRegressor.



 


\section{Experiments and Evaluation}
\label{sec:evaluation}
We created the topology shown in Figure~\ref{fig:topology} in ExoGENI to conduct our experiments. Among the $23$ nodes in this topology, we specify $6$ data origins and $6$ data sinks as the end hosts 
to transfer a batch of data files ($886$ files) of different sizes that we randomly acquired from OSG. The emulated topology follows the power law, i.e., $4$ routing 
nodes in the middle to emulate the backbone domains and the rest emulate the access domains in between the backbone nodes and the end hosts. 

We ran two sets of experiments to collect two sets of raw training data. In the first one, called {\it Partial},
data transfers only happen between the origins and sinks, where every origin node sends all the $886$ files 
to all the receiving nodes in parallel. In the second one, called {\it Complete}, data transfers happen between all the end host pairs. 
We further parallelized the data transfer process to reduce the emulation time down to about twenty-four hours for this particular network.  

The file integrity is being checked at the receiving end host and each file transfer accounts for one data flow and therefore a
 data sample in a training data set.

For each experiment, probabilistic integrity error or network impairment via the Chaos Jungle tool is injected to the $54$ link interfaces and $12$ end hosts in sequence with the given probability setting, which 
For each fault injection scenario, the entire set of {\it Partial} or {\it Complete} data transfers are conducted. Each link interface or node component with fault injected represents a label.  
The receiving node checks if a received file is identical to its original copy via checksum and marks this data transfer as a failure data sample if checksums do not match. 
We treat retransmission as a separate feature for the data samples.  A file could also be missed at the destination due to ultimate transfer failure which is also treated as a failure. 
If the checksum matches, this data transfer becomes a success data sample in the training data set. Otherwise, it is labeled by the corresponding faulty element, one of the total $66$ labels in this study. 
As a very basic benchmark, we note the accuracy of random classification would be merely $\frac{1}{66}$. 

Through extensive training data analysis, we hope to achieve three import goals in our RCA study:  (1) performance comparison in terms of accuracy and training time performance of different models, 
(2) impact quantification of different types of features on the model performance, esp. the file size and the transfer throughput, because the other features like data transfer source and destination, integrity error, 
and transfer status (retransmission, etc.) are all indispensable, (3) handling of the inherited data imbalance in the training data set.  

We compared different variants out of the three categories presented in Section~\ref{sec:ml} and presented the best model from each one: Random Forest, Linear SVC, and Multinomial Naive Bayes.
We use One Hot Encoding for all the categorical features. 

Each figure shows five different performance metrics under four scenarios. The five metrics are F1 score with per-flow inference, the accuracy with per-flow inference, and the $Top-1$, $Top-2$, and $Top-3$ accuracy with
aggregated flow inference. The four scenarios are {\it Partial} and the {\it Complete} data sets with {\it All} features (file and network) or only network ({\it No File}) features.

\begin{figure}[!ht]
\begin{center}
\includegraphics[width=0.45\textwidth]{./figure/rf-accuracy}
\end{center}
\caption{Classification Accuracy with Random Forest Model}
\label{fig:dt}
\end{figure}

Fig.~\ref{fig:dt} presents the results from training random forest models with different data sets. Two prominent observations stand out. 
First, it clearly shows that the model with {\it Complete} data performs significantly better than the {\it Partial} case. This illustrates the importance of sufficient 
coverage of the network path information in the training data set. It also shows that, even without the full coverage of network path between the network routers, 
the flows between end hosts along can guarantee very high RCA inference accuracy.   
Secondly, training with {\it All Features} outperforms its {\it No File Features} counterpart by a large margin. This means the file transfer statistics can boost the inference performance.  
We also learnt that the file transfer throughput has bigger impact than the file size, though the result is not shown here due to page limit. 

When we zoom into more details, we can see that the combined {\it Complete-All Features} data set presents satisfactory performance even with the single flow-base testing data samples with 
both F1 score and accuracy reaching above $0.90$. When the aggregated flow data samples are used for inference, the accuracy scores perfect $1$. The next best scenario is when {\it All Features} presented with 
{\it Partial} data transfer, while the single flow-based inference only achieved under $0.5$ accuracy, the aggregated flow-based inference doubles the accuracy and quickly narrows down the root cause to the 
top $2$ elements. However, without more path coverage, it doesn't achieve perfect accuracy until $Top-10$. For the next two scenarios, the aggregated flow inference helps achieve better performance 
and {\it Complete} path coverage appears more important than the inclusion of file transfer features. However, the best it can achieve is a mere $80\%$ $Top-3$ accuracy.  

\begin{figure}[!ht]
\begin{center}
\includegraphics[width=0.45\textwidth]{./figure/svc-accuracy}
\end{center}
\caption{Classification Accuracy with SVM Model}
\label{fig:svm}
\end{figure}

Fig.~\ref{fig:svm} depicts the results using a linear SVC model. It shows the same tendency with regards to {\it Complete} file transfer coverage and the aggregated flow inference. 
However, the inclusion of file transfer features poses very negative impacts. We believe this is due to the incapability of SVM family models in dealing with mixed numerical and categorical features. 
Overall, comparing to the rain forest model, it scores much worse in every performance metric as it only obtained less than $0.7$ $Top-3$ accuracy.

\begin{figure}[!ht]
\begin{center}
\includegraphics[width=0.45\textwidth]{./figure/nb-accuracy}
\end{center}
\caption{Classification Accuracy with Multinomial Naive Bayes}
\label{fig:bn}
\end{figure}

Fig.~\ref{fig:bn} shows the results with the Multinomial Naive Bayes model. Again, the accuracy increases with $k$. The single flow level accuracy is very poor and the accuracy increases dramatically with bigger $k$. 
It in general follows the pattern of the SVM model except it performs better in all the metrics. For example, its $Top-3$ accuracy is above $0.8$. Another difference is that, in the {\it Partial} scenario, it performs better with the {\it All-Features} than that with only {\it No File Features (NF) } for the F1-score, flow-based accuracy and the $Top-1$ accuracy.

We next look into the training time of the above three models. From Table.~\ref{tab:time}, it is clear that even the linear SVC model presented takes a substantially longer time than the other two to converge in all four cases. The SVC with the default RBF kernel takes a longer time and fails to converge after several hours even for the {\it Partial} data set. Between the other two, the BN model takes the shortest time (in sub-seconds) though all the Random Forest model trainings converge in a few seconds. The next observation is that the training with the {\it Complete} data set finishes much faster than that with {\it Partial} data set. This makes sense because more complete training data helps the model training converge faster. Similar observation follows regarding the features, where the data set with the {\it All Features (AF)} converges faster to some extent.

\begin{table}[!ht]
\caption{Training Time }
\label{tab:time}
%\vspace{-0.1in}
\begin{center}
\begin{tabular}{ |c|c|c|c|c| } 
 \hline
  & Partial-AF & Partial-NF & Complete-AF & Complete-NF\\ 
 \hline
 Forest & 5.45s & 2.42s & 4.23s & 1.54s\\ 
 \hline
 Bayes & 0.5s & 1.54s & 0.18s & 0.43s\\
 \hline
 SVC & 575s & 600s & 75s &70s\\ 
 \hline
\end{tabular}
\end{center}
%\vspace{-0.1in}
\end{table}
Combining both training accuracy and training time, the random forest model appears to be a clear winner for the RCA problem under our study since its trained model achieves excellent accuracy performance with rather short training time. BN models deserve further exploration due to its fast convergency and relative benign performance, especially networks of larger scale.

In order to gain more insight into the classification accuracy performance, we observed that the majority of mislabeled data in the classification inference are those with labels of end host faults in all the scenarios whose accuracy is less than $1$.  This is because the raw training data set is highly imbalanced due to the different impacts of the injected failures on the data files being transferred~\ref{sub:ml:imbalance}. There are significantly fewer integrity errors caused by the faulty end hosts, \ie, significantly less labeled data in these classes. 

We take the {\it Complete-No File Features} data set with the Random Forest model as an example. The details are shown in Table~\ref{tab:class}. We first categorize all the labeled data into two groups: those with faulty link interfaces and those with faulty end hosts, {\it Max} and {\it Min} represent the maximum and the minimum number of samples in a class in each category and the $Top-k$ up to $k=3$ accuracy under each category is calculated separately. We can see that there are a hundred times more samples in a few link failure classes than those in all the end host failure classes, and samples from some link failures classes are about ten times less than those from some other link failure classes. As a result, all the end host failures are misclassified in all the cases, while most link failure cases are correctly classified and the accuracy reaches $1$ when $k=3$.

\begin{table}[!ht]
\caption{Classification Accuracy Differentiation}
\label{tab:class}
%\vspace{-0.1in}
\begin{center}
\begin{tabular}{ |c|c|c|c|c| } 
 \hline
  \multicolumn{5}{|c|}{Link Faults} \\
 \hline
 Max & Min & $Top-1$ & $Top-2$ & $Top-3$\\ 
 \hline
 1443 & 121  & 0.70 &  0.96 & 1 \\
 \hline
\end{tabular}

\begin{tabular}{ |c|c|c|c|c|} 
 \hline
\multicolumn{5}{|c|}{Host Faults} \\
 \hline
 Max & Min & $Top-1$ & $Top-2$ & $Top-3$ \\ 
 \hline
11 & 11  & 0 &  0 & 0\\
  \hline
\end{tabular}
\end{center}
%\vspace{-0.1in}
\end{table}

The main techniques to solve the dataset imbalance problem are to rebalance the data via oversampling or downsampling data from different classes. Since the labeled data subsets from the end host failure classes are rather small, the pure downsampling techniques will not improve on the model performance. We therefore focus on two representative oversampling techniques: random and SMOTE as well as two methods that combine oversampling and downsampling.
These methods have been implemented in the {\it imbalanced-learn} library~\cite{imbalance-learn:web}.

The random oversampling approach is straightforward in which new samples are randomly generated by copying the existing samples from the underrepresented classes. The SMOTE(Synthetic Minority Oversampling Technique ) ~\cite{smote:2002} method and its variants are among the most popular oversampling techniques. SMOTE generates new samples by interpolation rather than duplication of existing samples using variants of k-Nearest Neighbors classifier method. At the end of the oversampling with above example, the augmented data set will have 1443 labeled data for each of the 66 classes. 

SMOTETomek is a combined method that uses SMOTE for Oversampling and removes Tomek links for downsampling. Tomek links identify two samples that are too close to each other. SMOTEENN is another popular method that combines over- and under-sampling using SMOTE and Edited Nearest Neighbors.

For the  {\it Complete-All Features} data set, our analysis indicated that both Random and combined SMOTE methods resulted similar perfect $Top-k$ performance but reduced flow-based F1 Score and accuracy slightly. SMOTE actually deteriorated all the metrics. The observations on the  {\it Complete-No File Features} data set are similar. Since the performance of the original model is already perfect for the former case, we only present the results on the two {\it Partial} data sets.


\begin{figure}[!ht]
\begin{center}
\includegraphics[width=0.45\textwidth]{./figure/partial-all-oversampling}
\end{center}
\caption{Oversampling with All Features}
\label{fig:dt}
\end{figure}

\begin{figure}[!ht]
\begin{center}
\includegraphics[width=0.45\textwidth]{./figure/partial-nofile-oversampling}
\end{center}
\caption{Oversampling with No File Features}
\label{fig:dt}
\end{figure}

The results show that the SMOTEENN method doesn't help the performance. The other three all improve the performance to some extent though at different scale, especially on the $Top--k$ accuracy measurement. The basic Random oversampling demonstrates the best performance improvement in most cases.  The main takeaway is that multiple sampling methods have to be evaluated in order to find the best one.


\section{Conclusions and Future Work}
\label{sec:future}
We developed a machine learning based RCA system that combines the information and capabilities from both application and infrastructure layers. 
The impacts of different combinations of numerical and categorical data features under different realistic network and measurement assumptions are quantified in an emulation environment. 
The analysis validated that high RCA accuracy to the device level can be achieved with an efficient machine learning model even only partial network and flow level measurement information are available.

For our future work, we plan to study networks of larger scale with different topology characteristics, finer tuned ML models, and possible integration with limited network monitoring information. we will further explore the multi-granularity classification framework for large networks. While we focused on single faulty component scenario for integrity errors in this study, we plan to expand to other failure modes and performance degradation RCA systems with multiple concurrent faults. We will also study stochastic approaches that leverages the probability distribution characteristics of the network failures.  


\section*{Acknowledgments}
This work is funded by NSF award OAC-1839900. Results were obtained using the ExoGENI testbed supported by NSF.

\bibliographystyle{IEEEtran}
\bibliography{iris_sl,iris}

%\newpage
%\begin{appendix} 
%\hfill \break

{\LARGE Demo Description}\\

In order to obtain sufficient amount of training and test data, while guaranteeing experimental repeatability and efficiency, we created a suite of software tools to automate emulation experiments in a Cloud testbed, which can automatically create a large-scale OSPF-enabled virtual network system, initiate data transfers between end hosts, and inject arbitrary integrity errors into the virtual router interfaces and end hosts.  

Our main design objectives is to automate the experiment creation, configuration, and data collection at any given scale. This not only guarantees repeatable experiments at different scales, but also makes data manipulation for ML model training for RCA much more efficient.   

We will demonstrate the three main components of the emulation and RCA analysis system. An emulation experiment in ExoGENI testbed is depicted in Fig.~\ref{fig:chaosjungle}.

\subsection{Network creation and configuration}
An arbitrary topology can be created with nodes in the form of virtual machines (VM) running customized images. We leverage the Postboot scripting capability offered by ExoGENI testbed to automate the network configuration.  

In our experimental topology, we use a set of end hosts to emulate the OSG data sources and sinks, a virtual network consisting of core routers emulating the backbone network service providers (Internet2, ESNet, etc) and access routers emulating the access network service providers (regional research and education networks and campus networks). All nodes are connected with virtual layer-2 links with certain throughput guarantee as the private data plane. An extra management plane interface is also available on every node, which provides remote access from the public Internet. The end hosts run an Ubuntu image with our Chaos Jungle fault injection tools. The routers run an Ubuntu image with the Zebra software router and our Chaos Jungle fault injection tools.

At the virtual node booting time, our customized Postboot scripts automate the following steps: (1) detect all the interfaces and their IP addresses, and all the neighboring nodes and the links; (2) create the configuration files for Zebra and OSPF daemons and start the routing control plane at the router nodes; (3) detect and add the default routes at the end host nodes.    

ExoGENI provides an API to launch any given experimental topology. On a successful execution, our experimental topology will be up and running with routing configured and network reachability established among all end hosts.   

\subsection{Data transfer and fault injection}
For every experiment, we attach a controller node that can reach all the nodes in the topology via the management plane interface. The controller is provided a Postboot script that automatically learns the experimental topology, creates a list of end hosts, routers, and links, and populates the end hosts with a set of files for the data transfer.

A user can log into the controller node, modify the experimental configuration files that define the data origin and sink nodes, the list of nodes to introduce storage integrity error, the list of network link interfaces to be disrupted by the Chaos Jungle tool, and the fault injection probability.

Then the experiment software can be started to inject the fault, transfer the data files from the origins, check the integrity at the sinks, and collect the data. This sequence is repeated for every fault specified in the configuration file.

\begin{figure}[!ht]
\begin{center}
\includegraphics[width=0.48\textwidth]{./figure/ChaosJungle}
\end{center}
\caption{Network Emulation and Integrity Error Injection in ExoGENI}
\label{fig:chaosjungle}
\end{figure}

\subsection{Data collection and analysis}
At the last step, all the raw data will be processed and stored in final result database files with predefined feature columns. Each database entry represents one data transfer with features of file name, file size, origin, sink, access router, integrity error or not, etc. However, the forwarding path is unknown as it is controlled by the routing control plane process. The final result is exported to a Jupyter notebook environment where machine learning based data analysis is performed.

\hfill \break

{\LARGE Demo Set-Up}\\

The default display set-up plus Internet access is sufficient for our demonstration needs as we only run the client at the demo site. The actual experiments will be run in the remote Cloud.


%\end{appendix} 

% That's all folks!
\end{document}
