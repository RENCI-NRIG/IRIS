As we discussed in Section~\ref{sec:introduction}, missing data is pervasive in reality due to lost or unavailable measurement data . 
It means that some samples, in the training set or the test set, have missing features. 
Using the example network in Fig.~\ref{fig:example} to illustrate, during a diagnosis time window, the application may not incur traffic between 
$c1$ and $d1$, or it never needs to transfer data between the origin sites, or the application measurement system may corrupt or loss some 
measurement data for some traffic flows. All these will lead 'holes' in the feature columns in the data sets. In the first two cases, 
entire feature columns will be missing in our model~\ref{eq:inverse}.

Missing data can be categorized into three types: (i) the data is missing completely at random (MCAR) if the missingness does not depend 
on any of the observed and unobserved variables, (ii) the data is missing at random (MAR) if the missingness is dependent only on the observed variables, 
(iii) the data is missing not at random (MNAR) if the missingness is neither MCAR nor MAR, \ie, the missingness depends on both observed variables and the 
unobserved variables. The majority of existing studies used the MCAR assumption~\cite{Yoon2018GAINMD}. 

There are many kinds of missing data recovery methods commonly used in the literature. These methods largely fall into three categories.

The {\it univariate} methods imputes values in a  feature dimension using only non-missing values in that feature dimension. It simply replaces the missed 
values with certain statistics of the non-missing values such as the zero, mean, median, mode, max, or min. 

The {\it multivariate} imputation algorithms use the entire set of available feature dimensions to estimate the missing values based on the assumption 
of correlations between the feature dimensions. Each feature with missing values is modeled as a function of other features, and therefore the imputation itself 
is modeled as a regression problem that are trained and used to estimate the imputation. In order to achieve the best performance, especially to avoid the 
overfitting from certain features, it is conducted in a series of regression iterations: at each step, a feature is used as the output of other features and the resulted model is used to estimate the missing feature. After all features are processed or the designated max iteration is reached, the results of the final estimation are used to impute 
the missing data.

Our prediction model in Eq.~(\ref{eq:inverse})  uses the path (flow) measurements as the input. It naturally fits the multivariate imputation approach 
because the path failures caused by a common component failure are correlated. In contrast, the existing models based on Eq.~(\ref{eq:linear}) use the 
component failure as the input variables that are independent from each other. The {\it multivariate} imputation does not seem to make sense.
The {\it univariate} method is deemed not applicable due to the sparse nature of the feature matrix and lack of reasonable explanation.

Most recently,  the Generative Adversarial Nets ({\it GAN)} framework has shown good performance to generate the missing data.
In this model, the generator’s goal is to accurately impute missing data, and the discriminator’s goal is to distinguish between observed and imputed 
components. The discriminator is trained to minimize the classification loss (when classifying which components were observed and 
which have been imputed), and the generator is trained to maximize the discriminator’s misclassification rate. Thus, these 
two networks are trained using an adversarial process~\cite{Yoon2018GAINMD,Awan2021ImputationOM}.

There are off-the-shelf libraries that support both univariate and multivariate imputations in popular software packages like R and 
Scikit-learn~\cite{JSSv045i03,10.1371/journal.pone.0254720}. The imputation can also be performed multiple times with different 
random number seeds to generate multiple imputations. This is important if statistical analysis is needed, \eg, in the medical domain. 

Most of existing missing data studies focus on minimizing the imputation errors of the data in the feature space. However the ultimate goal 
is the performance of the prediction models in after missing data being imputed.

Corresponding to our model in Eq.~(\ref{eq:inverse}), missing data will cause values of some $x_p$ to be null. 
The feature space is defined in a $|P|$-dimensional space $\mathbf{X} = \mathbf{X_1} \times \ldots \times \mathbf{X_{|P|}}$. 
Following the MCAR assumption 
on the missing data, we can define a mask vector $M = (M_1, \ldots, M_{|P|})$ taking random values in ${(0, 1)}^{|P|}$.  
A sample vector $X = (X_1 \times \ldots \times X_{|P|})$ 
can be masked by $M$ to generate a corresponding sample vector with missing data $\tilde{X} = \tilde{X}_1 \times \ldots \times \tilde{X}_{|P|}$ as follow:

\[
\tilde{X_p} = 
\begin{cases}
  X_p & \text{if $M_p = 1$} \\
  null & \text{otherwise}
\end{cases}
\]

From an arbitrary missing rate $r \in (0, 1)$, a random mask vector $M_r$ can be created to emulate missing data from a given feature matrix $X$. 
For a particular missing feature $X_r \in X$, the imputation essentially creates a regression model that makes $X_r$ the output variable and all the 
other features the input variables.   

\begin{flalign}\label{eq:imputation}
\begin{aligned}
X_r = F(x_1, \cdots, x_p, \cdots, x_{|P|} ), \ p \neq r \\
\end{aligned}
\end{flalign}

At the end of the imputation, a recovered data $\hat{X_r}$ is generated to goal is to make these as close as possible.

Our main results are based on the MCAR missing data model, the {\it multivariate} imputation algorithms, and regularized regression model, 
which can be summarized in the following pipeline definition with Scikit\_Learn.
%~\cite{Scikit:web}. 
\begin{verbatim}
    estimator = make_pipeline(
        IterativeImputer(random_state=0, 
        		missing_values=np.nan, 
        		estimator=impute_estimator),
        PolynomialFeatures(poly),
        br_estimator
    )
\end{verbatim}

In the pipeline, the {\it impute\_estimator} specifies the regressor for missing data imputation and the {\it br\_estimator} specifies the regressor to infer the 
localized failure probability. We added a {PolynomialFeatures} element to evaluate if polynomials of higher degree perform better than the linear regressor.

This pipeline construct allow us to systematically evaluate the performance of multiple regressors in both {\it impute\_estimator} and {\it br\_estimator}, as well 
as tuning their hyperparameters. As we discussed earlier, in theory, Lasso should be a suitable regressor in both places. We also evaluated other popular 
regressors that include Ridge, BayesianRidge, ExtraTreesRegressor, and KNeighborsRegressor.



 
